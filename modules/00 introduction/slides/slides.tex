\documentclass[10pt]{beamer}
\usetheme{metropolis}           % Use metropolis theme

\usepackage{appendixnumberbeamer}

\usepackage{pgfpages}
\setbeamertemplate{note page}[plain]
\setbeameroption{show notes on second screen=right}
\makeatletter 
% the following fixes bug which causes normal text to be white instead of
% template default when notes are enabled
% (see https://tex.stackexchange.com/questions/232168/normal-text-is-invisible-when-using-beamer-with-notes-and-xelatex)
\def\beamer@framenotesbegin{% at beginning of slide
     \usebeamercolor[fg]{normal text}
      \gdef\beamer@noteitems{}% 
      \gdef\beamer@notes{}% 
}
\makeatother

\usepackage{verbatim}

\usepackage[scale=3]{ccicons}   % creative commons icons

\title{CSCI 351 --- Principles of Parallel Computing}
\date{}
\author{Jeremy Iverson}
\institute{College of Saint Benedict \& Saint John's University}
\begin{document}
  \maketitle

  \begin{frame}{logistics}
    \begin{itemize}
      \item instructor(s)
        \begin{itemize}
          \item Jeremy Iverson (\href{mailto:jiverson002@csbsju.edu}{\nolinkurl{jiverson002@csbsju.edu}})
          \item PENGL 213, (320) 363-3083
          \item office hours: \begin{tabular}[t]{rrrcr}
                                MW & 12:45 & - & 13:45 \\
                                 T & 11:30 & - & 12:30
                              \end{tabular}
        \end{itemize}
      \item textbook
        \begin{itemize}
          \item \emph{An Introduction to Parallel Programming}, 1st Edition,
            Pacheco
        \end{itemize}
      \item website
        \begin{itemize}
          \item \url{https://csbsju.instructure.com/courses/7664}
        \end{itemize}
    \end{itemize}

    \note{I prefer to be called Jeremy\\}
    \note{Encourage questions right away\\}
    \note{Emphasize the importance of the Canvas site for finding information
      about the class}
  \end{frame}

  \begin{frame}{course objectives}
    \begin{itemize}
      \item understand common parallel architectures
      \item design and analyze parallel algorithms
      \item write efficient parallel programs
    \end{itemize}
  \end{frame}

  \begin{frame}{course content}
    \begin{itemize}
      \item parallel architectures
      \item C programming
      \item Pthreads
      \item OpenMP
      \item MPI
      \item performance analysis
      \item parallel algorithm design
      \item collective communication operations
      \item matrix algorithms
      \item sorting algorithms
      \item graph algorithms
    \end{itemize}

    \note{Ask students to point out some of the topics that they many have seen
      before and explain what they are\\}
    \note{Point out that each of these bullets corresponds to one of the
      ``modules'' on Canvas (except the last three, which are under advanced
      topics)}
  \end{frame}

  \begin{frame}[fragile]{assignments}
    \begin{itemize}
      \item four (4) assignments (serial, Pthreads/OpenMP, MPI, ???)
        \begin{itemize}
          \item all assignments are to be written in C or C\verb|++|
        \end{itemize}
      \item do your own work
      \item use each other as resources only
      \item due dates are strict (no partial credit for late assignments)
    \end{itemize}

    \note{Should be student's own work, may work together but must individual
      solutions\\}
    \note{Want to encourage using each other as resources, but do not want some
      students to rely on other students for answers}
  \end{frame}

  \begin{frame}{exams}
    \begin{itemize}
      \item two (2) in-class exams (closed book / note)
      \item one (1) in-class final exam (closed book / open note) [cumulative]
    \end{itemize}

    \note{Make-up exams are possible under certain circumstances}
  \end{frame}

  \begin{frame}{grading}
    \begin{itemize}
      \item point distribution
        \begin{itemize}
          \item assignments: 12.5\% each (??\%total)
          \item exams: 15\% each
          \item final: 20\%
        \end{itemize}
    \end{itemize}
  \end{frame}

  \begin{frame}{scholastic conduct}
    \begin{itemize}
      \item Work must be completed in a manner consistent with the College of
        Saint Benedict's \& Saint John's University's codes for academic honesty
        (read that \href{http://www.csbsju.edu/academics}{\nolinkurl{here}}
        $\rightarrow$ Academic Catalog $\rightarrow$ Academic Policies and
        Regulations $\rightarrow$ Rights and Responsibilities).
      \item Copying another's work, allowing (even negligently) others to copy
        your work, relying heavily on information that you found on the web, is
        cheating and grounds for penalties in accordance with the institutional
        policies.
      \item There will be an absolutely zero tolerance policy. Any copying
        issues that will arise will be automatically reported to the
        institution. 
    \end{itemize}
  \end{frame}

  \begin{frame}{office hours}
    \begin{itemize}
      \item stop by anytime
      \item \href{http://exchange.csbsju.edu/owa/calendar/JIVERSON002@CSBSJU.EDU/Calendar/calendar.html}{my calendar}
    \end{itemize}

    \note{Mention outlook calendar \& my home page\\}
    \note{For those unfamiliar with Outlook meetings, then they should schedule
      another way and we will go over this in meeting\\}
    \note{Mention not generally availability in the afternoons}
  \end{frame}

  \begin{frame}{teaching philosophy}
    Your job is to empower those you teach; when you do for them what they
    should be doing for themselves, you create dependency rather than
    empowerment.

    It is easy to give in to the frustration that results from seeing amazing
    possibilities for the people you are teaching, and you want it more for them
    than they want it for themselves. 

    Don’t give in to that frustration! \\
    \hfill \footnotesize{- Based on passage from ``Resisting Happiness'' by
      Matthew Kelly}

    \note{In other words, don't expect me to give you answers. My job is to help
      you discover the answers for yourself.}
  \end{frame}

  \begin{frame}{expectations}
    \begin{itemize}
      \item prepare for class (do the reading!)
      \item participate in class
      \item be respectful
    \end{itemize}

    \note{\begin{itemize}
      \item Don’t just read, by try your best to make sense of the material
      \item Emphasize that students are not just recipients of my knowledge ---
        they can shape the direction of the course
      \item Encourage students to stop and think before replying.
      \item Don’t be discouraged if comprehension is not apparent right away,
        but don’t be complacent
      \item \textbf{Seek help (TAs and me)}
      \item Students can expect the same things from me.
    \end{itemize}}
  \end{frame}

  \begin{frame}{serial algorithms}
    \begin{itemize}
      \item some algorithms that will be relevant to this course:
        \begin{itemize}
          \item finding min/max values in array
          \item matrix-vector and matrix-matrix multiplication
          \item Gaussian elimination
          \item depth-first, breadth-first, and best-first traversals
          \item minimum spanning tree
          \item single source shortest path
          \item quicksort, radix sort, bucket sort, counting sort, sample sort
          \item A* and IDA* heuristic search
        \end{itemize}
    \end{itemize}

    \note{\begin{itemize}
      \item You do not need to know them all now, but it is a good idea to have
        them on your radar
      \item Wikipedia is a good resource in this case
    \end{itemize}}
  \end{frame}

  \begin{frame}[standout]
    \usebeamercolor[white]{normal text} % override bug fix in preamble
    questions?
  \end{frame}

  \appendix

  \begin{frame}{notecards}
    \begin{itemize}
      \item given name
      \item preferred name and pronunciation
      \item anything you would like me to know about you
      \item at least one of the following:
        \begin{itemize}
          \item reason for taking the class
          \item what you are hoping to learn from the class
        \end{itemize}
    \end{itemize}
  \end{frame}

  \begin{frame}{activity}
    \begin{itemize}
      \item stand up
      \item think of the number 1
      \item repeat while there is at least one other person standing
        \begin{itemize}
          \item find another (just one) person standing
          \item whoever is thinking of a higher number, adds the other persons
            number to theirs
          \item whoever is thinking of a lower number, sits down
          \item in the event of a tie, the taller of the two people sits down
        \end{itemize}
    \end{itemize}

    \note{\begin{itemize}
      \item How many people did you talk to?
      \item What is the maximum number of people that you talked to as a
        function of the result?
    \end{itemize}}
  \end{frame}

  \begin{frame}
    \begin{center}\ccbysa\end{center}

    except where otherwise noted, this worked is licensed under
    \href{http://creativecommons.org/licenses/by-sa/4.0/}{creative commons
    attribution-sharealike 4.0 international license}
  \end{frame}
\end{document}
