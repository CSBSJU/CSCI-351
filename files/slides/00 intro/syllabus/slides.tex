\documentclass[10pt, t]{beamer}
\usetheme{metropolis}           % Use metropolis theme

\usepackage{appendixnumberbeamer}

\ifnotes
  \hypersetup{final}
  \usepackage{pgfpages}
  \setbeamertemplate{note page}[plain]
  \setbeameroption{show notes on second screen=right}
\fi

\usepackage{verbatim}

\usepackage[scale=3]{ccicons}   % creative commons icons

\title{CSCI 351 --- Principles of Parallel Computing}
\date{}
\author{Jeremy Iverson}
\institute{College of Saint Benedict \& Saint John's University}
\begin{document}
  \maketitle

  \begin{frame}{logistics}
    \begin{itemize}
      \item instructor(s)
        \begin{itemize}
          \item Jeremy Iverson (\href{mailto:jiverson002@csbsju.edu}{\nolinkurl{jiverson002@csbsju.edu}})
          \item MAIN 258, (320) 363-5542
          \item \emph{Zoom} office hours: TBD
        \end{itemize}
      \item textbook
        \begin{itemize}
          \item \emph{An Introduction to Parallel Programming}, 1st Edition,
            Pacheco
        \end{itemize}
      \item website
        \begin{itemize}
          \item \url{https://csbsju.instructure.com/courses/14840}
        \end{itemize}
      \item zoom
        \begin{itemize}
          \item \url{https://csbsju.zoom.us/j/91795643146?pwd=ZGFRejNkcTlzQ3FISktTb1V5aysxZz09}
        \end{itemize}
    \end{itemize}

    \note{I prefer to be called Jeremy\\}
    \note{I am married and my wife and I have five children, age six and under,
      the youngest is ~1 month old\\}
    \note{Encourage questions right away\\}
    \note{Emphasize the importance of the Canvas site for finding information
      about the class}
  \end{frame}

  \begin{frame}{zoom etiquette}
    \begin{itemize}
      \item all students should be logged into Zoom, including those in the
        classroom
      \item keep microphone muted unless you are talking
      \item use raise hand feature if you need the instructor's attention
      \item monitor chat window; use chat feature to:
        \begin{itemize}
          \item comment or add appropriate input during class activities, or
          \item ask questions that are not time sensitive.
        \end{itemize}
    \end{itemize}

    \note{I will address questions asked in chat, as time allows; questions
      which are not answered, I will address outside of class via Canvas
      announcement or discussions.\\}
  \end{frame}

  \begin{frame}[standout]
    zoom test

    \note{after zoom is setup and working for students, take a break until
      ?????\\}
  \end{frame}

  \begin{frame}{class structure}
    \begin{itemize}
      \item class will be organized around 25 minute blocks
      \item between each block, we will take a 5 minute break
        \pause
        \begin{itemize}
          \item because there are an abundance of breaks scheduled, please wait
            until a break to take care of things that require you leaving your
            seat, unless absolutely necessary
        \end{itemize}
    \end{itemize}

    \only<2->{a goal of this schedule is to reduce the impact of internal and
      external interruptions on focus and flow.}
  \end{frame}

  \begin{frame}{today's schedule}
    \begin{itemize}
      \item syllabus
      \item motivation for studying parallel computing
      \item challenges of parallel computing
      \item workflows for our high-performance cluster
      \item brief review of CPU and memory
    \end{itemize}
  \end{frame}

  \begin{frame}{course objectives}
    Presents the \textbf<3>{theoretical foundations of parallel computing} and
    \textbf<2>{an overview of several parallel computing models}. Exposes
    students to current parallel programming models and systems through
    \textbf<4>{projects}. Teaches students the ability to determine the most
    appropriate model for a given task.

    \note<2>{we will study the two most common parallel computing models used in
      practice, and briefly discuss a third}
    \note<3>{
      this will be structured in a similar way as CSCI 338 for those who have
      taken that course. although the material there will not be required for
      this class, it will be helpful. all that will be required here is a sound
      understanding of asymptotic analysis of functions (i.e., big-oh).
      \begin{itemize}
        \item first, we will study a framework for analyzing parallel
          algorithms
        \item then we will study techniques for decomposing a problem in order
          to formulate a parallel solution
      \end{itemize}
    }
    \note<4>{throughout the course, we will be applying what we are learning
      via a series of programming assignments}
  \end{frame}

  \begin{frame}[fragile]{assignments}
    \begin{itemize}
      \item four (4) assignments (serial, Pthreads/OpenMP, MPI, ???)
        \begin{itemize}
          \item all assignments are to be written in C or C\verb|++|
        \end{itemize}
      \item do your own work
      \item use each other as resources only
      \item due dates are strict (no partial credit for late assignments)
    \end{itemize}

    \note{Should be student's own work, may work together but must individual
      solutions\\}
    \note{Want to encourage using each other as resources, but do not want some
      students to rely on other students for answers}
  \end{frame}

  \begin{frame}{exams}
    \begin{itemize}
      \item two (2) in-class exams (closed book / closed note)
      \item one (1) in-class final exam (closed book / open note) [cumulative]
    \end{itemize}

    \note{Make-up exams are possible under certain circumstances}
  \end{frame}

  \begin{frame}{grading}
    \begin{itemize}
      \item point distribution
        \begin{itemize}
          \item assignments: 50\% total
          \item exams: 15\% each
          \item final: 20\%
        \end{itemize}
    \end{itemize}
  \end{frame}

  \begin{frame}{scholastic conduct}
    \begin{itemize}
      \item Work must be completed in a manner consistent with the College of
        Saint Benedict's \& Saint John's University's codes for academic honesty
        (read that \href{http://www.csbsju.edu/academics}{\nolinkurl{here}}
        $\rightarrow$ Academic Catalog $\rightarrow$ Academic Policies and
        Regulations $\rightarrow$ Rights and Responsibilities).
      \item Copying another's work, allowing (even negligently) others to copy
        your work, relying heavily on information that you found on the web, is
        cheating and grounds for penalties in accordance with the institutional
        policies.
      \item There will be an absolutely zero tolerance policy. Any copying
        issues that will arise will be automatically reported to the
        institution.
    \end{itemize}
  \end{frame}

  \begin{frame}{expectations}
    \begin{itemize}
      \item prepare for class (do the reading!)
      \item participate in class
      \item be respectful
    \end{itemize}

    \note{\begin{itemize}
      \item Don’t just read, by try your best to make sense of the material
      \item Emphasize that students are not just recipients of my knowledge ---
        they can shape the direction of the course
      \item Encourage students to stop and think before replying.
      \item Don’t be discouraged if comprehension is not apparent right away,
        but don’t be complacent
      \item Students can expect the same things from me.
    \end{itemize}}
  \end{frame}

  \begin{frame}{office hours}
    \begin{itemize}
      \item stop by anytime
      \item \href{http://exchange.csbsju.edu/owa/calendar/JIVERSON002@CSBSJU.EDU/Calendar/calendar.html}{my calendar}
    \end{itemize}

    \note{Mention outlook calendar \& my home page\\}
    \note{For those unfamiliar with Outlook meetings, then they should schedule
      another way and we will go over this in meeting\\}
    \note{Mention not generally availability in the mornings}
  \end{frame}

  %\begin{frame}{serial algorithms}
  %  \begin{itemize}
  %    \item some algorithms that will be relevant to this course:
  %      \begin{itemize}
  %        \item finding min/max values in array
  %        \item matrix-vector and matrix-matrix multiplication
  %        \item Gaussian elimination
  %        \item depth-first, breadth-first, and best-first traversals
  %        \item minimum spanning tree
  %        \item single source shortest path
  %        \item quicksort, radix sort, bucket sort, counting sort, sample sort
  %        \item A* and IDA* heuristic search
  %      \end{itemize}
  %  \end{itemize}

  %  \note{\begin{itemize}
  %    \item You do not need to know them all now, but it is a good idea to have
  %      them on your radar
  %    \item Wikipedia is a good resource in this case
  %  \end{itemize}}
  %\end{frame}

  \begin{frame}[standout]
    questions?

    \note{after answering questions, take a break until ?????\\}
  \end{frame}

  \appendix

%  \begin{frame}{notecards}
%    \begin{itemize}
%      \item given name
%      \item preferred name and pronunciation
%      \item anything you would like me to know about you
%      \item at least one of the following:
%        \begin{itemize}
%          \item reason for taking the class
%          \item what you are hoping to learn from the class
%        \end{itemize}
%    \end{itemize}
%  \end{frame}
%
%  \begin{frame}{activity}
%    \begin{itemize}
%      \item stand up
%      \item think of the number 1
%      \item repeat while there is at least one other person standing
%        \begin{itemize}
%          \item find another (just one) person standing
%          \item whoever is thinking of a higher number, adds the other persons
%            number to theirs
%          \item whoever is thinking of a lower number, sits down
%          \item in the event of a tie, the taller of the two people sits down
%        \end{itemize}
%    \end{itemize}
%
%    \note{\begin{itemize}
%      \item How many people did you talk to?
%      \item What is the maximum number of people that you talked to as a
%        function of the result?
%    \end{itemize}}
%  \end{frame}

  \begin{frame}[c]
    \begin{center}\ccbysa\end{center}

    except where otherwise noted, this worked is licensed under
    \href{http://creativecommons.org/licenses/by-sa/4.0/}{creative commons
    attribution-sharealike 4.0 international license}
  \end{frame}
\end{document}
